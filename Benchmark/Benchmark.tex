% !TeX spellcheck = en_US
\documentclass[a4paper]{scrartcl}

\usepackage[utf8]{inputenc}
\usepackage[english]{babel}
\usepackage[T1]{fontenc}
\usepackage{lmodern}
\usepackage{amsmath}
\usepackage{amssymb}
\usepackage{pdflscape}
\usepackage{geometry}
\usepackage{xcolor}
\usepackage{graphicx}
\usepackage{todonotes}
\usepackage{subfig}% multi-part figures with separate captions per part
\setlength{\parindent}{0pt}

\usepackage{biblatex}
\addbibresource{references.bib}


%\geometry{a4paper, top=25mm, left=30mm, right=20mm, bottom=30mm,
%headsep=10mm, footskip=12mm}

\newcommand{\itab}[1]{\hspace{0em}\rlap{#1}}
\newcommand{\tab}[1]{\hspace{.2\textwidth}\rlap{#1}}


\title{Benchmarking HDT and GraphRePair}
\author{Philip Frerk}
\date{\today}
 

\begin{document}
\maketitle


\section{Fragen an Michael}

\begin{itemize}
	\item wo soll ich erklären, dass GRP nicht immer gut mit $k^2$ zusammenarbeitet? (related work oder später)
\end{itemize}

\section{Experiments}

\begin{tabular}{|c|c|}
	\hline 
	\textbf{Ontology} & \textbf{Dictionary} \\ 
	\hline 
	Entity-Teilgraph (wenige Vorkommen der \\relevanten Properties) & blank nodes (Polygon-Daten) \\ 
	\hline 
	Teilgraph mit vielen Vorkommen & literals (DBPedia abstracts and SWDF) \\ 
	\hline 
\end{tabular} 
\\\\\\What the different datasets contain:\\\\
\begin{tabular}{|c|c|c|c|}
	\hline 
	& \textbf{symmetric} & \textbf{inverse} & \textbf{transitive} \\ 
	\hline 
	\textbf{DBPedia} & yes & yes & yes \\ 
	\hline 
	\textbf{Wordnet} & yes & (hypernym <-> hyponym) only for nouns & yes \\ 
	\hline 
\end{tabular} 
\\\\\\All together: persondata graph

\section{Dictionary vs Graph}

\begin{tabular}{|c|c|c|}
	\hline 
	\textbf{dict  / graph compr }& \textbf{yes} & \textbf{no} \\ 
	\hline 
	\textbf{yes} & Normal HDT or GRP+HDT-Dict & HDT with plain triples \\ 
	\hline 
	\textbf{no} & (HDT or GRP) with HDT's plain dictionary  & N-triples \\ 
	\hline 
\end{tabular} 
\\\\\\HDT plain triples: possible\\
HDT Plain dictionary: not possible

\section{Notes}

warum hypernym und hyponym nicht invers? (nur nomen Teil)
Wordnet: (hypernym und hyponym inverse), antonym symmetrisch

open data: URIs aufrufen



DBPEdia: persondata (nur personen rausnehmen: da kommen die relevanten Properties vor)

Eine person nehmen und von dort aus teilgraphen erzeugen

persondata erweitern mit abstracts und birthplace koord.






\end{document}