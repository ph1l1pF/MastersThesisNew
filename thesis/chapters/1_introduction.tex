% !TeX spellcheck = en_US
\chapter{Introduction}\label{ch:introduction}

\section{Motivation}
The majority of data on the Internet is unstructured, because it is mainly text. That makes it difficult for machines to extract knowledge from the data and answer specific queries. In the context of the Semantic Web, an attempt is made to build a knowledge base using structured data. A possible framework for structured data is the \ac{RDF}\footnote{https://www.w3.org/RDF/}, in which triples of the form $ (subject, predicate, object) $ are used in order to express certain facts. A set of triples naturally forms a graph, whereas the different $subjects$ and $objects$ are nodes and the $predicates$ are edges of that graph. As those graphs express facts, they are called knowledge graphs.

In reality, such knowledge graphs can become very large with millions or even billions of nodes and edges. The following three use cases can then become very hard or even infeasible: storage, transmission and processing. One way to circumvent this problem is using compression. There are many existing compressors which work for all kinds of data, but the problem is that the compressed data is not query-able. Also, their compression might not be so strong, since they do not take advantage of the special features of RDF. For these reasons RDF-specific compression techniques are of interest. An example of such an RDF compressor is \ac{HDT} from~\cite{hdt}. Here, the data is made smaller by a compact representation of the triples. 

There is a completely different compression technique which is called grammar-based compression. This method has so far been tested very little for RDF graphs. As the name suggests, it is based on the principle of a formal grammar, with productions that can be nested among each other. There are not yet many grammar-based compression algorithms for graphs. One example is \ac{GRP}~\cite{maneth}. This is a compressor that works for any type of graph and is therefore also applicable for RDF.

In this thesis, it will be investigated to what extent grammar-based compression is suitable for RDF and whether it delivers even better results than HDT.


\section{General Idea}

As already mentioned, there are essentially three use cases for RDF data:

\begin{enumerate}
	\item Storage
	\item Transmission
	\item Processing/Consumption
\end{enumerate}

The idea is to make all these use cases easier / possible by compression, especially if very large amounts of data have to be handled. In the first two use cases, compression can be helpful by reducing the size of the data. Thus the data can be stored more compactly and above all can be transferred faster, which occurs frequently in reality and can become a problem due to possibly slow transfer speeds.

The third use case (Processing/Consumption) is about reading or writing access to data. The more common case of read access is typically the execution of queries on RDF data. These are most often neighborhood queries. A compression algorithm can help in this respect by making it possible to answer such queries directly on the compressed graphs. This may even be faster than on the original data due to the reduced size.

As stated in~\cite{maneth}, a graph compressed by GRP is not well suited for those just mentioned neighborhood queries. Such queries will take much longer than on the original graph. Therefore the thesis will focus on GRP's potential of compressing RDF data strongly and the query times will not be evaluated.



\section{Thesis Structure}

In Ch.~\ref{ch:related_work} the necessary fundamentals are presented. It will mainly be about the two compressors HDT and GRP. 

Ch.~\ref{ch:approach} deals with the approaches for the different aspects of the thesis. There, they will only be presented in a theoretical manner whereas in Ch.~\ref{ch:implementation} the implementation details will be presented.

In Ch.~\ref{ch:evaluation}, the approaches to the different tasks will be evaluated using real RDF data in order to confirm the theoretical hypotheses stated before.

Finally, the results of the thesis are summarized in Ch.~\ref{ch:summary_and_discussion} and the future work on this topic is presented.








