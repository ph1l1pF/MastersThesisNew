% !TeX spellcheck = de_DE
\documentclass[a4paper]{scrartcl}

\usepackage[utf8]{inputenc}
\usepackage[english]{babel}
\usepackage[T1]{fontenc}
\usepackage{lmodern}
\usepackage{amsmath}
\usepackage{amssymb}
\usepackage{pdflscape}
\usepackage{geometry}
\usepackage{xcolor}
\usepackage{graphicx}
\usepackage{todonotes}
\setlength{\parindent}{0pt}

\usepackage{biblatex}
\addbibresource{references.bib}


%\geometry{a4paper, top=25mm, left=30mm, right=20mm, bottom=30mm,
%headsep=10mm, footskip=12mm}

\newcommand{\itab}[1]{\hspace{0em}\rlap{#1}}
\newcommand{\tab}[1]{\hspace{.2\textwidth}\rlap{#1}}


\title{Understanding Maneth}
\author{Philip Frerk}
\date{\today}
 

\begin{document}
\maketitle

\section{Node IDs}

\begin{itemize}
	\item Die Node IDs werden nicht explizit gespeichert, wenn komprimiert wird. D.h. sie 'verschwinden', wenn ein Knoten durch das Anwenden eines Musters wegfällt
	\item Sie können durch eine vordefinierte Reihenfolge beim Ersetzen der Vorkommen eines Musters wiederhergestellt werden
	\item Kanten werden nach lexikographischer Reihenfolge ihrer Labels besucht
	
	Wenn die lexikographische Reihenfolge zweier Kanten gleich ist, dann wird die lex. Reihenfolge der Attachments genommen (wenn die auch gleich ist, dann ist auch die Reihenfolge egal).
\end{itemize}

\begin{tabular}{|c|c|}
	\hline 
	Schon gemacht & Noch nicht gemacht  \\ 
	\hline 
	RDF Compression (Specific properties en,  Types ru, Types es, Types de with en\\, Identica, Jamendo)&  Andere Datasets (zB Semantic web dog food), Unterschiedliche Komprresionsraten in den Datasets, Gründe untersuchen?\\ 
	\hline 
	&  \\ 
	\hline 
	&  \\ 
	\hline 
	&  \\ 
	\hline 
	&  \\ 
	\hline 
	&  \\ 
	\hline 
	&  \\ 
	\hline 
	&  \\ 
	\hline 
	&  \\ 
	\hline 
\end{tabular} 







\end{document}